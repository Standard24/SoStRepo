\subsection{Charakterisierung des Pr:YLF-Kristalls}

\subsubsection{Absorptionsspektrum}

Zur Bestimmung des Absorptionsspektrums des Pr:YLF-Kristalls wurde mit dem USB-Spektrometer ein
Referenzspektrum aufgezeichnet (Tageslicht vom Himmel), dann das Transmissionsspektrum des Kristalls
mit dem gleichen Licht aufgezeichnet und (vom Spektrometer) das Absorptionsspektrum als Differenz
berechnet.
Abb.~\ref{img:AbsSpec} zeigt das berechnete Spektrum.
Die Lage der Absorptionsmaxima ist eingezeichnet und in Tab.~ \ref{tab:AbsSpec} aufgeführt.

\begin{figure}[H]
\begin{center}
  \includegraphics[width=\textwidth]{AbsSpec.pdf}
  \caption{Absorptionsspektrum des Pr:YLF-Kristalls und Position der deutlichen Maxima.}
  \label{img:AbsSpec}
\end{center}
\end{figure}

\begin{table}[htb]
\caption{Positionen und relative Intensitäten der Absorptionsmaxima im Spektrum des
Pr:YLF-Kristalls.}
\begin{center}
\begin{tabular}{|c|c|}
\hline
Wellenlänge / nm & Absorption / b.E. \\ \hline
402 & 3918.9 \\ \hline
442 & 10287.3 \\ \hline
459 & 4653.3 \\ \hline
467 & 6203.5 \\ \hline
479 & 4147.4 \\ \hline
585 & 1947.5 \\ \hline
595 & 1420.8 \\ \hline
886 & 850.9 \\ \hline
935 & 535.6 \\ \hline
\end{tabular}
\end{center}
\label{tab:AbsSpec}
\end{table}


\subsubsection{Emissionsspektrum}

\paragraph{Aufbau und Durchführung}

\begin{figure}[H]
\begin{center}
  \includegraphics[width=\textwidth]{Aufbau3.pdf}
  \caption{Schematische Darstellung des für die Messung des Emissionsspektrums sowie der Lebensdauer des angeregten Zustands verwendeten Aufbaus. Der kollimierte Laserstrahl wird mit einer zweiten Linse (FL) mit einer Brennweite von 60\,mm auf den Kristall fokussiert, sodass dort eine möglichst große Intensität erzielt wird. Eine optische Faser wird so montiert, dass sie auf den Kristall gerichtet ist und an ein Spektrometer der Firma Lasertack angeschlossen. Das Signals des Spektrometers wird an einem Computer mit dem Programm \"Check Tr 9 5\" aufgenommen. Für die Messung der Lebensdauer wird das blaue Licht des Pumplasers über einen GG495 Filter geblockt und nur das vom Kristall emittierte Licht transmittiert. Das Fluoreszenzlicht des Kristalls wird dahinter mit einem Photodetektor gemessen und über eine Widerstandsbox auf einem Oszilloskop angezeigt. Das Oszilloskop ist zum triggern zusätzlich mit dem Diodenlaser Controller verbunden.}
  \label{img:aufbau3}
\end{center}
\end{figure}




\paragraph{Auswertung}

blablabla

\begin{figure}[H]
\begin{center}
  \includegraphics[width=\textwidth]{EmSpec.pdf}
  \caption{Emissionsspektrum des Pr:YLF-Kristalls und Position der deutlichen Maxima.}
  \label{img:EmSpec}
\end{center}
\end{figure}

\begin{table}[htb]
\caption{Positionen und relative Intensitäten der Emissionsmaxima im Spektrum des
Pr:YLF-Kristalls.}
\begin{center}
\begin{tabular}{|c|c|c|c|}
\hline
 & gem. Wellenlänge / nm & Emission / b.E. & th. Wellenlänge / nm \cite{Versuchsanleitung} \\ \hline
1 & 442 & 2168.6 & Pumplaserlicht \\ \hline
2 & 480 & 15633.9 & 480 ($^3$P$_0$ $\rightarrow$ $^3$H$_4$) \\ \hline
3 & 522 & 7761.3 & 523 ($^3$P$_1$ $\rightarrow$ $^3$H$_5$) \\ \hline
4 & 546 & 3115.6 &  \\ \hline
5 & 587 & 1099.0 &  \\ \hline
6 & 604 & 11627.5 & 607 ($^3$P$_0$ $\rightarrow$ $^3$H$_6$) \\ \hline
7 & 639 & 6787.0 & 640 ($^3$P$_0$ $\rightarrow$ $^3$F$_2$) \\ \hline
8 & 670 & 635.9 &  \\ \hline
9 & 697 & 1162.8 & 698 ($^3$P$_0$ $\rightarrow$ $^3$F$_3$) \\ \hline
10 & 720 & 1457.7 & 721 ($^3$P$_0$ $\rightarrow$ $^3$F$_4$) \\ \hline
11 & 962 & 508.6 &  \\ \hline
12 & 1047 & 366.7 &  \\ \hline
\end{tabular}
\end{center}
\label{tab:EmSpec}
\end{table}


\FloatBarrier


\subsubsection{Messung der Lebensdauer des angeregten Zustands}

\paragraph{Aufbau und Durchführung}

Für diesen Versuchsteil wird der in Abb.~\ref{img:aufbau3} dargestellte Aufbau verwendet.
Der Pumplaser wird hierbei im modulierten Bereich bei 710\,mA betrieben und das Oszilloskop darüber getriggert. Das Signal des Photodetektors geht über eine Widerstandsbox, welche bei 1k$\Omega$ betrieben wird, ebenfalls auf das Oszilloskop.

\paragraph{Auswertung}
Abb.~\ref{img:Lifetime} zeigt die Modulation der Laserspannung während der Messung und das
dazugehörige Fluoreszenzsignal, das von der Photodiode geliefert wird.
Das Photodiodensignal eines einzelnen Abschaltvorgangs ist auf Abb.~\ref{img:LifetimeFit} zu sehen. 
Als Fehler auf die Diodenspannung wird von der Spannungsauflösung des Oszilloskops
(0,2\,mV) ausgegangen und diese - unter Annahme einer Gleichverteilung der Fehler - durch
$2\sqrt{3}$ geteilt.
Der Fit des Signals erfolgt mit einer exponentiellen Abnahme mit der Zeitkonstanten~$\tau$ und dem
Untergrund~$U$. Das Fitergebnis ist auf Tab.~\ref{tab:Fit_lifetime} zu sehen.


\begin{figure}[H]
\begin{center}
  \includegraphics[width=.7\textwidth]{lifetime.png}
  \caption{Modulation des Lasers mit einem Rechtecksignal (gelb) zur Bestimmung der Lebensdauer der
  Fluoreszenz (blau, Messung als Spannungssignal der Photodiode).}
  \label{img:Lifetime}
\end{center}
\end{figure}


\begin{figure}[H]
\begin{center}
  \includegraphics[width=.9\textwidth]{lifetime.pdf}
  \caption{Exponentieller Fit des Spannungssignals der Photodiode $U_{\text{PD}}$ zur Bestimmung der
  Lebenszeit des angeregten Zustands. Der Fitbereich ist gelb markiert, auf die Darstellung der geringen Fehler
  wurde verzichtet.}
  \label{img:LifetimeFit}
\end{center}
\end{figure}

\begin{table}[htb]
\caption{Ergebnisse des Fits der Fluoreszenzlebensdauer mit
$y=A\,\exp(-x/\tau)\,+\,U$.}
\begin{center}
\begin{tabular}{|c|c|}
\hline
$A$ & 55.65\,$\pm$\,0.21\,mV \\ \hline
$\tau$ & 39.01\,$\pm$\,0.10\,\textmu s \\ \hline
$U$ & 0.605\,$\pm$\,0.008\,mV \\ \hline
\textchi$^2$ & 527.875 \\ \hline
\textchi$^2$/\,DoF & 0.450021 \\ \hline
\end{tabular}
\end{center}
\label{tab:Fit_lifetime}
\end{table}


\subsubsection{Messung der absorbierten Leistung}

\paragraph{Aufbau und Durchführung}

\begin{figure}[H]
\begin{center}
  \includegraphics[width=\textwidth]{Aufbau2.pdf}
  \caption{Schematische Darstellung des für die Messung der absorbierten Leistung verwendeten Aufbaus. Hierfür entfernen wir den Filter wieder und ersetzten den Photodetektor mit dem Powerhead welcher an das Powermeter angeschlossen wird.}
  \label{img:aufbau2}
\end{center}
\end{figure}

Wir verwenden hier den in Abb.~\ref{img:aufbau2} dargestellten Aufbau. Für 400, 600 und 800\,mA wurden jeweils bei 25\grad und 35\grad die Leistungen mit und ohne Kristall im Strahlengang gemessen.



\paragraph{Auswertung}
blabla
 
\begin{table}[htb]
\caption{Leistung am Leistungsmesskopf ohne Kristall im Strahlengang ($P_\text{ohne}$),
mit Kristall ($P_\text{mit}$), absorbierte Leistung ($P_\text{abs}$) und relative Absorption
$P_\text{abs}/P_\text{ohne}$ in Abhängigkeit von Lasertemperatur $T$ und Laserstrom $I$.}
\begin{center}
\begin{tabular}{|c|c|c|c|c|c|}
\hline
T / \grad & I / mA & $P_\text{mit}$ / mW & $P_\text{ohne}$ / mW & $P_\text{abs}$ / mW & Absorption / \% \\ \hline
25 & 400 & 114 & 427 & 313 & 73.3 \\ \hline
25 & 600 & 149 & 737 & 588 & 79.8 \\ \hline
25 & 800 & 87.4 & 532 & 444.6 & 83.6 \\ \hline
35 & 400 & 77.1 & 408 & 330.9 & 81.1 \\ \hline
35 & 600 & 105 & 718 & 613 & 85.4 \\ \hline
35 & 800 & 30.6 & 522 & 491.4 & 94.1 \\ \hline
\end{tabular}
\end{center}
\label{tab:Absorption}
\end{table}
