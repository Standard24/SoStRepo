\section{Zusammenfassung und Diskussion}

\subsection{Charakterisierung des Pumplasers}

Bei der Charakterisierung des Pumplasers wurde eine Erhöhung der Laserschwelle von
\begin{equation*}
b_{25\text{\grad}} = 124.7\pm1.0\,\text{mA}
\end{equation*}
bei 25\grad Betriebstemperatur auf
\begin{equation*}
b_{35\text{\grad}} = 133.0\pm1.0\,\text{mA}
\end{equation*}
bei 35\grad festgestellt.
Die Effizienz nimmt von
\begin{equation*}
a_{25\text{\grad}} = 1.730\pm0.005\,\text{mW\,/\,mA}
\end{equation*}
auf
\begin{equation*}
a_{35\text{\grad}} = 1.699\pm0.005\,\text{mW\,/\,mA}
\end{equation*}
ab.
Diese Messwerte entsprechen unseren Erwartungen.

Die Chiquadrat/DoF-Werte der beiden Fits liegen bei 1.6 und 0.7.
Die Abweichung vom Wert 1 kommt vermutlich daher,
dass die Fehler auf die einzelnen Messwerte aus der beobachteten Schwankung der Anzeige des
Leistungsmesskopfes abgeschätzt wurden und diese Abschätzung sehr ungenau war.
Eine genauere Analyse der Messungenauigkeit des Leistungsmesskopfes hätte hier durchgeführt werden
können, war aber für das Versuchsziel nicht relevant.
Der Fit der Kennlinie im Bereich zwischen 150\,mA und 700\,mA mit einer linearen Funktion erscheint
durch die Chiquadrat/DoF-Werte also gerechtfertigt.

Eine interessante Beobachtung ist das Überkreuzen der beiden Kennlinien bei 900\,mA;
über diesem Wert hat der Laser bei 35\grad eine höhere Leistung als bei 25\grad.

\subsection{Charakterisierung des Pr:YLF-Kristalls}

\subsubsection{Absorptionsspektrum}

Die Wellenlängen der gemessenen Linien liegen ca. 10\,nm über den Literaturwerten für Pr$^{3+}$.
Diese systematische Abweichung könnte dadurch verursacht werden,
dass das Praseodym nicht in Reinform, sondern in den Pr:YLF-Kristall eingebettet vorliegt.

Die erste Absorptionsspitze bei 402\,nm lässt sich keinem Übergang von Praseodym zuordnen.
Die Form der Spitze (asymmetrisch, langsamer Abfall am rechten Rand) deutet darauf hin,
dass es sich hier um ein Artefakt durch den Untergrund handelt und nicht um eine Linie.

Die Linien 7-9 können ebenfalls nicht Praseodym zugeordnet werden.
Es könnte sich hier um Absorptionen von Li$^+$, Y$^{3+}$ oder F$^-$ handeln.



\FloatBarrier

\subsubsection{Emissionsspektrum}

Das Emissionsspektrum ist deutlich komplexer als das Absorptionsspektrum,
da der angeregte Zustand $^3$P$_2$ über viele verschiedene Niveaus in den Grundzustand übergehen
kann.
Einige Linien werden daher nicht zugeordnet.
Bei der ersten Linie handelt es sich entweder um gestreutes Licht des Pumplasers oder um den
direkten Übergang aus dem angeregten Zustand in den Grundzustand.

Die Übereinstimmung unserer Messdaten mit den Literaturwerten ist gut,
die größte Abweichung der Daten ist 3\,nm.



\subsubsection{Messung der Lebensdauer des angeregten Zustands}

Für die Lebensdauer des angeregten Zustands erhalten wir
\begin{equation*}
\tau = 39.01\,\pm\,0.10\,\text{\textmu s}.
\end{equation*}
Dieser Wert liegt deutlich unterhalb des Wertes von 50\,\textmu s,
der in der Versuchsanleitung angegeben ist.
Der Grund dafür, dass wir eine schnellere Relaxation messen, ist uns nicht klar.

Der Chiquadrat/DoF-Wert von 0.45 rechtfertigt den exponentiellen Fit und deutet darauf hin,
dass der Fehler auf die Photodiodenspannung zu groß abgeschätzt wurde.

\subsubsection{Messung der absorbierten Leistung}

Auf Tab.~\ref{tab:Absorption} ist eine starke Abhängigkeit der absorbierten Leitung von
Betriebstemperatur und Betriebsstrom des Pumplasers zu erkennen.
Die Absorption beträgt zwischen 73.3\,\% und 94.1\,\%.
Ursache für die Änderung der Absorption ist vermutlich die Änderung der Wellenlänge des Pumplasers
mit Strom und Temperatur und damit ein mehr oder weniger großer Überlapp mit dem Übergang
$^3$H$_4 \rightarrow\, ^3$P$_2$.
Die Wellenlänge ändert sich bei einer Temperaturänderung von 10\,K um 0.5\,nm und bei einer
Stromänderung von 0.4\,A um 1.32\,nm.
Die beiden Änderungen besitzen ungefähr die selbe Größenordnung,
was tendenziell mit den Messdaten übereinstimmt.
Eine quantitative Analyse ist hier aber schwierig,
da die Messwerte vom Leistungsmesskopf schwer reproduzierbar waren und zeitweise stark schwankten. 


\subsection{Charakterisierung des Lasers auf der roten Linie}

\subsubsection{Aufnahme der Kennlinien}

Bei der Charakterisierung des roten Lasers wurden folgende Laserschwellen gemessen:
\begin{equation*}
b_{25\text{\grad}} = 204.9\pm2.3\,\text{mA}
\end{equation*}
\begin{equation*}
b_{35\text{\grad}} = 177.7\pm1.9\,\text{mA}
\end{equation*}
Die Werte liegen weit unter dem Zielwert von 320\,mA, der in der Versuchsanleitung angegeben ist.
Das Absinken der Laserschwelle mit steigender Temperatur kann über die Zunahme der Absorption im
Kristall erklärt werden (Tab.~\ref{tab:Absorption}),
wodurch schon bei geringerem Strom ausreichend Leistung für den Laserbetrieb zur Verfügung steht.

Für die Effizienzen wurden folgende Werte erhalten:
\begin{equation*}
a_{25\text{\grad}} = 11.91\pm0.10\,\text{\textmu W\,/\,mA}
\end{equation*}
\begin{equation*}
a_{35\text{\grad}} = 9.77\pm0.07\,\text{\textmu W\,/\,mA}
\end{equation*}
Einen geringen Beitrag zur Abnahme mit steigender Temperatur leistet der Pumplaser.
Für die relativ große Differenz der beiden Werte besitzen wir keine direkte Erklärung.
Allerdings sollten bei der Interpretation der Fitergebnisse für Effizienzen und
Laserschwellen die hohen Chiquadrat/DoF-Werte berücksichtigt werden (8.8 und 11.3).
Da die Justierung der Komponenten im Strahlengang mit dem Ziel einer möglichst hohen Laserleistung
durchgeführt wurde und nicht mit einem stabilen Betriebszustand des Lasers,
führte der zeitweise instabile Zustand des Lasers zu großen Schwankungen bei der Leistungsmessung
und die Kennlinie kann deswegen nur schlecht mit einem linearen Fit beschrieben werden.


\subsubsection{Messung der Laserleistung mit Photodiode}



\subsubsection{Erzeugung verschiedener Moden}


\subsubsection{Messung des dynamischen Verhaltens}


\subsection{Wellenlängenselektion}

\subsubsection{Selektion über Reflektivität der Spiegel}

Ähnlich wie bei der Aufnahme der Kennlinie für den roten Laser wurde der Aufbau bei dem grünen und
gelben Laser auf eine Maximierung der Laserleistung hin optimiert.
Die maximal erreichten Leistungen sind 12.7\,mW für den grünen Laser und 8.6\,mW für den gelben.
Auf einen Fit der Kennlinien wurde verzichtet,
da ein linearer Verlauf nur sehr schwach erkennbar ist.
Der Laserbetrieb setzt bei dem grünen Laser bei 750\,mA bzw. 800\,mA Betriebsstrom ein,
beim gelben Laser bei 850\,mA.
Diese Laserschwellen sind deutlich höher als bei dem roten Laser,
weil die Wirkungsquerschnitte der Übergänge geringer sind.

\subsubsection{Selektion über doppelbrechenden Kristall}

\subsubsection{Selektion mit Littrow-Prisma}
