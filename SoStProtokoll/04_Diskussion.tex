\section{Zusammenfassung und Diskussion}

\subsection{Charakterisierung des Pumplasers}

Bei der Charakterisierung des Pumplasers wurde eine Erhöhung der Laserschwelle von
$124.7\pm1.0$\,mA bei 25\grad Betriebstemperatur auf $133.0\pm1.0$\,mA bei 35\grad festgestellt.
Die Effizienz nimmt von $1.730\pm0.005$\,mW\,/\,mA auf $1.699\pm0.005$\,mW\,/\,mA ab.
Diese Messwerte entsprechen unseren Erwartungen.

Die Chiquadrat/DoF-Werte der beiden Fits liegen bei 1.6 und 0.7.
Die Abweichung vom Wert 1 kommt vermutlich daher,
dass die Fehler auf die einzelnen Messwerte aus der beobachteten Schwankung der Anzeige des
Leistungsmesskopfes abgeschätzt wurden und diese Abschätzung sehr ungenau war.
Eine genauere Analyse der Messungenauigkeit des Leistungsmesskopfes hätte hier durchgeführt werden
können, war aber für das Versuchsziel nicht relevant.
Der Fit der Kennlinie im Bereich zwischen 150\,mA und 700\,mA mit einer linearen Funktion erscheint
durch die Chiquadrat/DoF-Werte also gerechtfertigt.

Eine interessante Beobachtung ist das Überkreuzen der beiden Kennlinien bei 900\,mA;
über diesem Wert hat der Laser bei 35\grad eine höhere Leistung als bei 25\grad.

\subsection{Charakterisierung des Pr:YLF-Kristalls}

\subsubsection{Absorptionsspektrum}

Die Zuordnung einiger Linien des Absorptionsspektrums ist mit Hilfe von spektroskopischen Daten
(Tab.~\ref{tab:AbsSpecTh}) möglich.
Die Linien 2-6 können so fünf Übergängen aus dem Grundzustand zugeordnet werden
(Tab.~\ref{tab:AbsSpecVgl}).
Die systematische Abweichung der Wellenlängen um ca. 10\,nm könnte dadurch verursacht werden,
dass das Praseodym nicht in Reinform, sondern in den Pr:YLF-Kristall eingebettet vorliegt.

Die erste Absorptionsspitze bei 402\,nm lässt sich keinem Übergang von Praseodym zuordnen.
Die Form der Spitze (asymmetrisch, langsamer Abfall am rechten Rand) deutet darauf hin,
dass es sich hier um ein Artefakt handelt und nicht um eine Linie.

Die Linien 7-9 können ebenfalls nicht Praseodym zugeordnet werden.
Es könnte sich hier um Absorptionen von Li$^+$ oder F$^-$ handeln.


\begin{table}[htb]
\caption{Übergänge aus dem Grundzustand 3H4 in angeregte Zustände von Pr$^{3+}$ \cite{NIST_ASD}.}
\begin{center}
\begin{tabular}{|c|c|c|}
\hline
Zielniveau & Wellenzahl / cm$^{-1}$ & Wellenlänge / nm \\ \hline
$^3$H$_5$ & 2152.09 & 4646.65 \\ \hline
$^3$H$_6$ & 4389.09 & 2278.38 \\ \hline
$^3$F$_2$ & 4996.61 & 2001.36 \\ \hline
$^3$F$_3$ & 6415.24 & 1558.79 \\ \hline
$^3$F$_4$ & 6854.75 & 1458.84 \\ \hline
$^1$G$_4$ & 9921.24 & 1007.94 \\ \hline
$^1$D$_2$ & 17334.4 & 576.888 \\ \hline
$^3$P$_0$ & 21389.8 & 467.512 \\ \hline
$^3$P$_1$ & 22007.5 & 454.391 \\ \hline
$^3$P$_2$ & 23160.6 & 431.768 \\ \hline
$^1$I$_6$ & 22211.5 & 450.216 \\ \hline
\end{tabular}
\end{center}
\label{tab:AbsSpecTh}
\end{table}

\begin{table}[htb]
\caption{Zuweisung der gemessenen Absorptionslinien zu angeregten Zuständen von~Pr$^{3+}$.}
\begin{center}
\begin{tabular}{|c|c|c|}
\hline
  & gem. Wellenlänge / nm & th. Wellenlänge / nm \\ \hline
1 & 402 &  \\ \hline
2 & 442 & 432 (3P2) \\ \hline
3 & 459 & 450 (1I6) \\ \hline
4 & 467 & 454 (3P1) \\ \hline
5 & 479 & 468 (3P0) \\ \hline
6 & 585 & 577 (1D2) \\ \hline
7 & 595 &  \\ \hline
8 & 886 &  \\ \hline
9 & 935 &  \\ \hline
\end{tabular}
\end{center}
\label{tab:AbsSpecVgl}
\end{table}

\FloatBarrier

\subsubsection{Emissionsspektrum}

\begin{table}[htb]
\caption{Zuweisungen der Linien des gemessenen Emissionsspektrums zu den Übergängen von~Pr$^{3+}$.}
\begin{center}
\begin{tabular}{|c|c|c|}
\hline
  & gem. Wellenlänge / nm & th. Wellenlänge / nm \cite{Versuchsanleitung} \\ \hline
1 & 442 & Artefakt durch Pumplaser \\ \hline
2 & 480 &  \\ \hline
3 & 522 & 523 (3P1 $\rightarrow$ 3H5) \\ \hline
4 & 546 &  \\ \hline
5 & 587 &  \\ \hline
6 & 604 & 607 (3P0 $\rightarrow$ 3H6) \\ \hline
7 & 639 & 640 (3P0 $\rightarrow$ 3F2) \\ \hline
8 & 670 &  \\ \hline
9 & 697 &  \\ \hline
10 & 720 & 721 (3P0 $\rightarrow$ 3F4) \\ \hline
11 & 962 &  \\ \hline
12 & 1047 &  \\ \hline
\end{tabular}
\end{center}
\label{tab:EmSpecVgl}
\end{table}

\subsubsection{Messung der Lebensdauer des angeregten Zustands}

Diskussion chiquadrat von Lebensdauermessung, Vergleich von Lebensdauer mit Literaturwert

höhere Absorption durch Verstimmung des Pumplasers mit Temperatur und Strom?

\subsubsection{Messung der absorbierten Leistung}

\subsection{Charakterisierung des Lasers auf der roten Linie}
Hohe Chiquadrate erklären. Instabiler Betrieb, Modensprünge..?


\subsection{Wellenlängenselektion}