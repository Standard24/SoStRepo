\section{Zusammenfassung und Diskussion}

\subsection{Charakterisierung des Pumplasers}

Bei der Charakterisierung des Pumplasers wurde eine Erhöhung der Laserschwelle von
\begin{equation*}
b_{25\text{\grad}} = 124.7\pm1.0\,\text{mA}
\end{equation*}
bei 25\grad Betriebstemperatur auf
\begin{equation*}
b_{35\text{\grad}} = 133.0\pm1.0\,\text{mA}
\end{equation*}
bei 35\grad festgestellt.
Die Effizienz nimmt von
\begin{equation*}
a_{25\text{\grad}} = 1.730\pm0.005\,\text{mW\,/\,mA}
\end{equation*}
auf
\begin{equation*}
a_{35\text{\grad}} = 1.699\pm0.005\,\text{mW\,/\,mA}
\end{equation*}
ab.
Diese Messwerte entsprechen unseren Erwartungen.

Die Chiquadrat/DoF-Werte der beiden Fits liegen bei 1.6 und 0.7.
Die Abweichung vom Wert 1 kommt vermutlich daher,
dass die Fehler auf die einzelnen Messwerte aus der beobachteten Schwankung der Anzeige des
Leistungsmesskopfes abgeschätzt wurden und diese Abschätzung sehr ungenau war.
Eine genauere Analyse der Messungenauigkeit des Leistungsmesskopfes hätte hier durchgeführt werden
können, war aber für das Versuchsziel nicht relevant.
Der Fit der Kennlinie im Bereich zwischen 150\,mA und 700\,mA mit einer linearen Funktion erscheint
durch die Chiquadrat/DoF-Werte also gerechtfertigt.

Eine interessante Beobachtung ist das Überkreuzen der beiden Kennlinien bei 900\,mA;
über diesem Wert hat der Laser bei 35\grad eine höhere Leistung als bei 25\grad.

\subsection{Charakterisierung des Pr:YLF-Kristalls}

\subsubsection{Absorptionsspektrum}

Die Zuordnung einiger Linien des Absorptionsspektrums ist mit Hilfe von spektroskopischen Daten
(Tab.~\ref{tab:AbsSpecTh}) möglich.
Die Linien 2-6 können so fünf Übergängen aus dem Grundzustand zugeordnet werden
(Tab.~\ref{tab:AbsSpecVgl}).
Die systematische Abweichung der Wellenlängen um ca. 10\,nm könnte dadurch verursacht werden,
dass das Praseodym nicht in Reinform, sondern in den Pr:YLF-Kristall eingebettet vorliegt.

Die erste Absorptionsspitze bei 402\,nm lässt sich keinem Übergang von Praseodym zuordnen.
Die Form der Spitze (asymmetrisch, langsamer Abfall am rechten Rand) deutet darauf hin,
dass es sich hier um ein Artefakt handelt und nicht um eine Linie.

Die Linien 7-9 können ebenfalls nicht Praseodym zugeordnet werden.
Es könnte sich hier um Absorptionen von Li$^+$, Y$^{3+}$ oder F$^-$ handeln.


\begin{table}[htb]
\caption{Übergänge aus dem Grundzustand $^3$H$_4$ in angeregte Zustände von Pr$^{3+}$
\cite{NIST_ASD}.}
\begin{center}
\begin{tabular}{|c|c|c|}
\hline
Zielniveau & Wellenzahl / cm$^{-1}$ & Wellenlänge / nm \\ \hline
$^3$H$_5$ & 2152.09 & 4646.65 \\ \hline
$^3$H$_6$ & 4389.09 & 2278.38 \\ \hline
$^3$F$_2$ & 4996.61 & 2001.36 \\ \hline
$^3$F$_3$ & 6415.24 & 1558.79 \\ \hline
$^3$F$_4$ & 6854.75 & 1458.84 \\ \hline
$^1$G$_4$ & 9921.24 & 1007.94 \\ \hline
$^1$D$_2$ & 17334.4 & 576.888 \\ \hline
$^3$P$_0$ & 21389.8 & 467.512 \\ \hline
$^3$P$_1$ & 22007.5 & 454.391 \\ \hline
$^3$P$_2$ & 23160.6 & 431.768 \\ \hline
$^1$I$_6$ & 22211.5 & 450.216 \\ \hline
\end{tabular}
\end{center}
\label{tab:AbsSpecTh}
\end{table}

\begin{table}[htb]
\caption{Zuweisung der gemessenen Absorptionslinien zu Übergängen in angeregte Zustände
von~Pr$^{3+}$.}
\begin{center}
\begin{tabular}{|c|c|c|}
\hline
  & gem. Wellenlänge / nm & th. Wellenlänge / nm \\ \hline
1 & 402 & Artefakt \\ \hline
2 & 442 & 432 ($^3$P$_2$) \\ \hline
3 & 459 & 450 ($^1$I$_6$) \\ \hline
4 & 467 & 454 ($^3$P$_1$) \\ \hline
5 & 479 & 468 ($^3$P$_0$) \\ \hline
6 & 585 & 577 ($^1$D$_2$) \\ \hline
7 & 595 &  \\ \hline
8 & 886 &  \\ \hline
9 & 935 &  \\ \hline
\end{tabular}
\end{center}
\label{tab:AbsSpecVgl}
\end{table}

\FloatBarrier

\subsubsection{Emissionsspektrum}

\begin{table}[htb]
\caption{Zuweisungen der Linien des gemessenen Emissionsspektrums zu den Übergängen von~Pr$^{3+}$.}
\begin{center}
\begin{tabular}{|c|c|c|}
\hline
  & gem. Wellenlänge / nm & th. Wellenlänge / nm \cite{Versuchsanleitung} \\ \hline
1 & 442 & Pumplaserlicht \\ \hline
2 & 480 & 480 ($^3$P$_0$ $\rightarrow$ $^3$H$_4$) \\ \hline
3 & 522 & 523 ($^3$P$_1$ $\rightarrow$ $^3$H$_5$) \\ \hline
4 & 546 &  \\ \hline
5 & 587 &  \\ \hline
6 & 604 & 607 ($^3$P$_0$ $\rightarrow$ $^3$H$_6$) \\ \hline
7 & 639 & 640 ($^3$P$_0$ $\rightarrow$ $^3$F$_2$) \\ \hline
8 & 670 &  \\ \hline
9 & 697 & 689? ($^3$P$_0$ $\rightarrow$ $^3$F$_3$) \\ \hline
10 & 720 & 721 ($^3$P$_0$ $\rightarrow$ $^3$F$_4$) \\ \hline
11 & 962 &  \\ \hline
12 & 1047 &  \\ \hline
\end{tabular}
\end{center}
\label{tab:EmSpecVgl}
\end{table}

In Tab.~\ref{tab:EmSpecVgl} werden einige Linien des Emissionsspektrums mit Übergängen von Pr$^{3+}$
identifiziert.
Das Emissionsspektrum ist deutlich komplexer als das Absorptionsspektrum,
da der angeregte Zustand $^3$P$_2$ über viele verschiedene Niveaus in den Grundzustand übergehen
kann.
Einige Linien werden daher nicht zugeordnet.
Bei der ersten Linie handelt es sich entweder um gestreutes Licht des Pumplasers oder um den
direkten Übergang aus dem angeregten Zustand in den Grundzustand.

Die Übereinstimmung unserer Messdaten mit den Literaturwerten ist gut,
nur bei der 9. Linie besteht eine größere Abweichung (8\,nm).



\subsubsection{Messung der Lebensdauer des angeregten Zustands}

Für die Lebensdauer des angeregten Zustands erhalten wir
\begin{equation*}
\tau = 39.01\,\pm\,0.10\,\text{\textmu s}.
\end{equation*}
Dieser Wert liegt deutlich unterhalb des Wertes von 50\,\textmu s,
der in der Versuchsanleitung angegeben ist.
Der Grund dafür, dass wir eine schnellere Relaxation messen, ist uns nicht klar.

Der Chiquadrat/DoF-Wert von 0.45 rechtfertigt den exponentiellen Fit und deutet darauf hin,
dass der Fehler auf die Photodiodenspannung zu groß abgeschätzt wurde.

\subsubsection{Messung der absorbierten Leistung}

Auf Tab.~\ref{tab:Absorption} ist eine starke Abhängigkeit der absorbierten Leitung von
Betriebstemperatur und Betriebsstrom des Pumplasers zu erkennen.
Die Absorption beträgt zwischen 73.3\,\% und 94.1\,\%.
Ursache für die Änderung der Absorption ist vermutlich die Änderung der Wellenlänge des Pumplasers
mit Strom und Temperatur und damit ein mehr oder weniger großer Überlapp mit dem Übergang
$^3$H$_4 \rightarrow\, ^3$P$_2$.
Die Wellenlänge ändert sich bei einer Temperaturänderung von 10\,K um 0.5\,nm und bei einer
Stromänderung von 0.4\,A um 1.32\,nm.
Die beiden Änderungen besitzen ungefähr die selbe Größenordnung,
was tendenziell mit den Messdaten übereinstimmt.
Eine quantitative Analyse ist hier aber schwierig,
da die Messwerte vom Leistungsmesskopf schwer reproduzierbar waren und zeitweise stark schwankten. 


\subsection{Charakterisierung des Lasers auf der roten Linie}
Hohe Chiquadrate erklären. Instabiler Betrieb, Modensprünge..?


\subsection{Wellenlängenselektion}