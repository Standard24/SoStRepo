\section{Zusammenfassung und Diskussion}

\subsection{Charakterisierung des Pumplasers}

Bei der Charakterisierung des Pumplasers wurde eine Erhöhung der Laserschwelle von
$124.7\pm1.0$\,mA bei 25\grad Betriebstemperatur auf $133.0\pm1.0$\,mA bei 35\grad festgestellt.
Die Effizienz nimmt von $1.730\pm0.005$\,mW\,/\,mA auf $1.699\pm0.005$\,mW\,/\,mA ab.
Diese Messwerte entsprechen unseren Erwartungen.

Die Chiquadrat/DoF-Werte der beiden Fits liegen bei 1.6 und 0.7.
Die Abweichung vom Wert 1 kommt vermutlich daher,
dass die Fehler auf die einzelnen Messwerte aus der beobachteten Schwankung der Anzeige des
Leistungsmesskopfes abgeschätzt wurden und diese Abschätzung sehr ungenau war.
Eine genauere Analyse der Messungenauigkeit des Leistungsmesskopfes hätte hier durchgeführt werden
können, war aber für das Versuchsziel nicht relevant.
Der Fit der Kennlinie im Bereich zwischen 150\,mA und 700\,mA mit einer linearen Funktion erscheint
durch die Chiquadrat/DoF-Werte also gerechtfertigt.

Eine interessante Beobachtung ist das Überkreuzen der beiden Kennlinien bei 900\,mA;
über diesem Wert hat der Laser bei 35\grad eine höhere Leistung als bei 25\grad.

\subsection{Charakterisierung des Pr:YLF-Kristalls}

Zuweisung der Linien von Absorptionsspektrum und Emissionsspektrum zu den Übergängen 

Diskussion chiquadrat von Lebensdauermessung, Vergleich von Lebensdauer mit Literaturwert

höhere Absorption durch Verstimmung des Pumplasers mit Temperatur und Strom?

\subsection{Charakterisierung des Lasers auf der roten Linie}
Hohe Chiquadrate erklären. Instabiler Betrieb, Modensprünge..?


\subsection{Wellenlängenselektion}