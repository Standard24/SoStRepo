\section{Physikalische Grundlagen}

\subsection{Funktionsweise eines Lasers}

1. Describe the general working principles of a laser. In your answer, explain both the fundamental
physical background and the optical arrangement. If several laser transitions are possible, which
criteria determine the actual output wavelength?

\subsection{Stabilitätskriterium des Resonators}

5. What are the stability conditions of a cavity? To determine the stability range of the concentric
cavity used for second harmonic generation (Sec. 2.3.3), make a plot of g 1 g 2 (see Eq. 2.3.3) as a
function of the mirror spacing L. In your calculation, assume a concentric cavity which consists of
a spherical mirror (R 1 = 100 mm) and a spherical mirror (R 2 = 150 mm). Compare these results
to the stability curve of a hemispherical cavity consisting of a spherical mirror (R 1 = 100 mm) and
a flat mirror.



\subsection{Eigenschaften des Pr:YLF-Lasers}

2. What are the specific characteristics of a Pr:YLF laser, i.e., why is it interesting to use a Pr:YLF
laser compared to other solid state lasers? In your answer, also make a comparison of the Pr:YLF
laser with the ruby laser and the Nd:YAG laser.


\subsection{Wellenlängenselektion}


3. Compare the working principles of a birefringent tuner with those of a Littrow prism.


\subsection{Frequenzverdopplung}

4. How does second harmonic generation work and why does it have a low
efficiency?



\subsection{Laserschutzbrillen}


7. Based on the spectra shown in Fig. 8, explain why YLW laser safety glasses are used in the
experiment.