\subsection{Charakterisierung des Lasers auf der roten Linie}

\subsubsection{Aufbau}

\begin{figure}[H]
\begin{center}
  \includegraphics[width=\textwidth]{Aufbau4.png}
  \caption{Schematische Darstellung des Aufbaus des Lasers mit roter Linie. Der Aufbau ist zunächst wie in ~\ref{img:aufbau3}, allerdings benötigen wir das Spektrometer nicht und wir bauen zusätzlich einen Resonator ein.
   Direkt vor dem Kristall wird ein planer Spiegel und in knapp 10\,cm zu diesem hinter dem Kristall ein Spiegel mit einem Krümmungsradius von 10\,cm platziert. Beide Spiegel sind hoch transmittierend bei 445\,nm und hochreflektiv beschichtet bei 580-720\,nm.}
  \label{img:aufbau4}
\end{center}
\end{figure}

Für die Charakterisierung des Lasers auf der roten Linie verwenden wir den in Abb.~\ref{img:aufbau4} dargestellten Aufbau. Die Spiegel wurden nach dem Stabilitätskriterium mit einem Abstand von kleiner 10\,cm zuneinander um den Kristall aufgebaut. Die Spiegel wurden so justiert, dass die Rückreflexe auf dem einfallenden Strahl liegen und nachdem durch rotieren des Kristalls rotes Laserlicht sichtbar war wurde ihre Position auf maximale Ausgangsleistung optimiert. Die Position des Kristalls wurde anschließend ebenfalls auf maximale Ausgangsleistung durch drehen, kippen sowie verschieben im Fokus des Pumplasers optimiert. 
Anschließend wurde zunächst mit dem OP-2 VIS Powerhead (bis 30\,mW) die Laserleistung für 25\grad C und 35\grad C jeweils von 0 bis 1.4\,A in 50\,mA Schritten gemessen.
Anschließend wurde mit dem Photodetektor das Ausgangssignal bei 600\,mA und 25\grad C bestimmt. Zusätzlich wurde das Ausgangssignal auf einem weißen Schirm in einem Abstand von ca. 1\,m betrachtet und dabei der hintere Resonatorspiegel leicht verkippt und mehrere Fotos der verschiedenen Lasermoden aufgenommen. 
Zur Bestimmung des dynamischen Laserverhaltens wurde das Signal mit dem Photodetektor im modulierten Betrieb des Pumplasers bei 710\,mA und 25\grad C aufgenommen. Das Signal wurde hier über 128 Perioden gemittelt.




\subsubsection{Aufnahme der Kennlinien}


\begin{figure}[H]
\begin{center}
  \includegraphics[width=\textwidth]{PI_rot.pdf}
  \caption{PI-Kennlinie des roten Lasers bei 25\grad und 35\grad
  Betriebstemperatur. Der modulationsfreie Bereich wurde mit $y=a(x-b)$
  gefittet, um die Laserschwelle~$b$ und die Effizienz~$a$ zu bestimmen.}
  \label{img:PI_rot}
\end{center}
\end{figure}


\begin{table}[htb]
\caption{Ergebnisse der Fits der PI-Kennlinien roten Lasers mit $y=a(x-b)$ von 200\,mA bis
700\,mA Laserstrom.}
\begin{center}
\begin{tabular}{|c|c|}
\hline
\textbf{25\grad} &  \\ \hline
$a$ & 11.91\,$\pm$\,0.10\,\textmu W\,/\,mA \\ \hline
$b$ & 204.9\,$\pm$\,2.3\,mA \\ \hline
\textchi$^2$ & 79.2873 \\ \hline
\textchi$^2$/\,DoF & 8.8097 \\ \hline
 &  \\ \hline
\textbf{35\grad} &  \\ \hline
$a$ & 9.77\,$\pm$\,0.07\,\textmu W\,/\,mA \\ \hline
$b$ & 177.7\,$\pm$\,1.9\,mA \\ \hline
\textchi$^2$ & 101.847 \\ \hline
\textchi$^2$/\,DoF & 11.3164 \\ \hline
\end{tabular}
\end{center}
\label{tab:Fits_PI_rot}
\end{table}

\FloatBarrier

\subsubsection{Messung der Laserleistung mit Photodiode}



\subsubsection{Erzeugung verschiedener Moden}


\subsubsection{Messung des dynamischen Verhaltens}

\begin{figure}[H]
\begin{center}
  \includegraphics[width=.7\textwidth]{LaserEinschwingvorgang.png}
  \caption{Photodiodenspannung (blau) nach Einschalten der Laserspannung (gelb).
  Ein dynamisches Einschwingen der Laserintensität ist erkennbar.}
  \label{img:Einschwingen}
\end{center}
\end{figure}